%%%%%%%%%%%%%%%%%%%%%%%%%%%%%%%%%%%%%%%%%%%%%%%%%%%%%%%%%%%%%%%%%%%%%%%%%%%%%%%
% Chapter 4: HtmlForm renderer
%%%%%%%%%%%%%%%%%%%%%%%%%%%%%%%%%%%%%%%%%%%%%%%%%%%%%%%%%%%%%%%%%%%%%%%%%%%%%%%

%++++++++++++++++++++++++++++++++++++++++++++++++++++++++++++++++++++++++++++++

objetivos
desarrollo
resultados
problemas encontrados y soluciones


%++++++++++++++++++++++++++++++++++++++++++++++++++++++++++++++++++++++++++++++
\section{Creaci\'on del renderer HtmlForm}
\label{3:sec2}

Este renderer permite generar un documento HTML5 con un formulario en el que se encuentran todas las
preguntas listas para ser completadas desde el navegador. A continuaci\'on se enumerar\'an todas sus 
caracter\'{\i}sticas:

\begin{itemize}
  \item A\~{n}adida la opci\'on de a\~{n}adir uno o m\'as JavaScripts al cuestionario que se generar\'a.
  \item A\~{n}adida la opci\'on de a\~{n}adir uno o m\'as ficheros de fragmentos de c\'odigo HTML en la cabecera 
  del cuestionario que se generar\'a.
  \item A\~{n}adida la opci\'on de a\~{n}adir m\'as de una hoja de estilo CSS al cuestionario que se generar\'a.
  \item A\~{n}adida la opci\'on de a\~{n}adir un header y un footer personalizado al cuestionario que se generar\'a.
  Se puede especificar como un \textit{string} o indicar el path donde se encuentra el fichero que contiene 
  dicho c\'odigo HTML.
  \item Los textos de las preguntas admiten ahora caracteres HTML escapados.
  \item El cuestionario es capaz de renderizar expresiones escritas en {\bfseries LaTeX}.
  
  [foto]
  
  \item Las preguntas de completar permiten ahora respuestas n\'umericas y de c\'odigo JavaScript.
  
  [foto]
  
  \item Las espacios para rellenar las respuestas de las preguntas de completar se ajustan al tama\~{n}o de dicha respuesta.
  
  [foto]
  
  \item Dos nuevas maneras simplificadas de escribir preguntas de completar.
  
  [foto]
  
  \item Se han a\~{n}adido dos nuevos tipos de preguntas:
  \begin{itemize}
    \item Preguntas de {\bfseries Drag and Drop}: para preguntas de completar y preguntas tipo test (de respuesta \'unica
    y multirrespuesta)
    
    [foto]
    \bigskip
    
    \item Preguntas de {\bfseries programaci\'on}:
    \begin{itemize}
      \item Este tipo de preguntas s\'olo admite c\'odigo JavaScript debido a que la correcci\'on de preguntas tambi\'en 
      tiene lugar en el navegador cliente.
      \item Se crear\'a un \textit{textarea} de unas dimensiones definidas por defecto que tambi\'en se pueden personalizar
      y se colorear\'a el c\'odigo escrito para facilitar su lectura.
      \item La respuesta asignada a este tiempo de preguntas debe ser un c\'odigo JavaScript que valide la respuesta introducida
      por el alumno. Este c\'odigo puede escribirse en notaci\'on de \textit{string} o especificar el path donde se encuentra el
      fichero que contiene dicho c\'odigo.
      
      [foto]
    \end{itemize}
  \end{itemize}
  
  \item Validaci\'on autom\'atica de respuestas mediante JavaScript que muestra la nota obtenida al instante.
  \item Almacenamiento local de las respuestas introducidas usando {\bfseries Local Storage} de HTML5
  \item Men\'u contextual al hacer click derecho sobre el campo de respuesta para ver la respuesta correcta de dicha pregunta.
  Esta funcionalidad s\'olo est\'a disponible para preguntas de completar, cuyas respuestas sean \textit{strings},
  \textit{regexps} o num\'ericas.
  
  \bigskip
  [foto]
  \bigskip
  
  Para las preguntas tipo test existe un bot\'on que marca las respuestas correctas.
  
  [foto]
  
  \item Internacionalizaci\'on: la gema comprueba el idioma del sistema para ofrecer la traducci\'on adecuada al idioma del usuario
  Actualmente solo soporta ingl\'es y espa\~{n}ol. Para cualquier otro idioma, se utiliza el ingl\'es por defecto.
  
  [foto]

\end{itemize}

Adem\'as, se han creado test para comprobar el funcionamiento del renderer (usando Spec) y el cuestionario generado, usando Mocha, Chai y Karma.

%---------------------------------------------------------------------------------
% \section{Problemas encontrados y soluciones}
% \label{sec:2}

% \subsection{Correcci\'on de preguntas de Ruby en JavaScript}
% \label{subsec:2.3}
% \bigskip
% 
% bla, bla, bla
% 