%%%%%%%%%%%%%%%%%%%%%%%%%%%%%%%%%%%%%%%%%%%%%%%%%%%%%%%%%%%%%%%%%%%%%%%%%%%%%%%
% Chapter 3: Resultados
%%%%%%%%%%%%%%%%%%%%%%%%%%%%%%%%%%%%%%%%%%%%%%%%%%%%%%%%%%%%%%%%%%%%%%%%%%%%%%%

%++++++++++++++++++++++++++++++++++++++++++++++++++++++++++++++++++++++++++++++


Tras explicar en el cap\'{\i}tulo anterior la metodolog\'{\i}a empleada, los problemas encontrados
durante la fase de desarrollo e implementaci\'on y las soluciones halladas, a continuaci\'on se
detallar\'an todos los resultados obtenidos durante la fase de desarrollo del Trabajo de Fin de Grado: 
por un lado se encuentran las mejoras realizadas al DSL original de la gema y por otro, el enriquecimiento
de la misma con dos nuevos \textit{renderers}.

%++++++++++++++++++++++++++++++++++++++++++++++++++++++++++++++++++++++++++++++
\section{Mejoras del DSL original}
\label{3:sec1}

Fruto del estudio del c\'odigo y de ejecuciones sucesivas de la gema, se ha mejorado el funcionamiento
de la misma efectuando diversos cambios:

\begin{itemize}
  \item Correcci\'on de errores en el funcionamiento de la gema (enumerados en el cap\'{\i}tulo 2).
  \item Refactorizaci\'on de c\'odigo.
  \item A\~{n}adido manejo de excepciones tras errores de ejecuci\'on.
  \item A\~{n}adida la opci\'on en l\'inea de comandos \textit{--version} para comprobar la versi\'on de la gema. 
  \item A\~{n}adida la opci\'on en l\'inea de comandos \textit{--help} para ver la ayuda. 
\end{itemize}

%++++++++++++++++++++++++++++++++++++++++++++++++++++++++++++++++++++++++++++++
\section{Creaci\'on del renderer HTMLForm}
\label{3:sec2}

Este renderer permite generar un documento HTML5 con un formulario en el que se encontran todas las
preguntas listas para ser completadas desde el navegador. A continuaci\'on se enumerar\'an todas las sus 
caracter\'{\i}sticas:

\begin{itemize}
  \item A\~{n}adida la opci\'on de a\~{n}adir uno o m\'as JavaScripts al cuestionario que se generar\'a.
  \item A\~{n}adida la opci\'on de a\~{n}adir uno o m\'as ficheros de fragmentos de c\'odigo HTML en la cabecera 
  del cuestionario que se generar\'a.
  \item A\~{n}adida la opci\'on de a\~{n}adir m\'as de una hoja de estilo CSS al cuestionario que se generar\'a.
  \item A\~{n}adida la opci\'on de a\~{n}adir un header y un footer personalizado al cuestionario que se generar\'a.
  Se puede especificar como un \textit{string} o indicar el path donde se encuentra el fichero que contiene 
  dicho c\'odigo HTML.
  \item Los textos de las preguntas admiten ahora caracteres HTML escapados.
  \item El cuestionario es capaz de renderizar expresiones escritas en {\bfseries LaTeX}.
  
  [foto]
  
  \item Las preguntas de completar permiten ahora respuestas n\'umericas y de c\'odigo JavaScript.
  
  [foto]
  
  \item Las espacios para rellenar las respuestas de las preguntas de completar se ajustan al tama\~{n}o de dicha respuesta.
  
  [foto]
  
  \item Dos nuevas maneras simplificadas de escribir preguntas de completar.
  
  [foto]
  
  \item Se han a\~{n}adido dos nuevos tipos de preguntas:
  \begin{itemize}
    \item Preguntas de {\bfseries Drag and Drop}: para preguntas de completar y preguntas tipo test (de respuesta \'unica
    y multirrespuesta)
    
    [foto]
    \bigskip
    
    \item Preguntas de {\bfseries programaci\'on}:
    \begin{itemize}
      \item Este tipo de preguntas s\'olo admite c\'odigo JavaScript debido a que la correcci\'on de preguntas tambi\'en 
      tiene lugar en el navegador cliente.
      \item Se crear\'a un \textit{textarea} de unas dimensiones definidas por defecto que tambi\'en se pueden personalizar
      y se colorear\'a el c\'odigo escrito para facilitar su lectura.
      \item La respuesta asignada a este tiempo de preguntas debe ser un c\'odigo JavaScript que valide la respuesta introducida
      por el alumno. Este c\'odigo puede escribirse en notaci\'on de \textit{string} o especificar el path donde se encuentra el
      fichero que contiene dicho c\'odigo.
      
      [foto]
    \end{itemize}
  \end{itemize}
  
  \item Validaci\'on autom\'atica de respuestas mediante JavaScript que muestra la nota obtenida al instante.
  \item Almacenamiento local de las respuestas introducidas usando {\bfseries Local Storage} de HTML5
  \item Men\'u contextual al hacer click derecho sobre el campo de respuesta para ver la respuesta correcta de dicha pregunta.
  Esta funcionalidad s\'olo est\'a disponible para preguntas de completar, cuyas respuestas sean \textit{strings},
  \textit{regexps} o num\'ericas.
  
  \bigskip
  [foto]
  \bigskip
  
  Para las preguntas tipo test existe un bot\'on que marca las respuestas correctas.
  
  [foto]
  
  \item Internacionalizaci\'on: la gema comprueba el idioma del sistema para ofrecer la traducci\'on adecuada al idioma del usuario
  Actualmente solo soporta ingl\'es y espa\~{n}ol. Para cualquier otro idioma, se utiliza el ingl\'es por defecto.
  
  [foto]

\end{itemize}

Adem\'as, se han creado test para comprobar el funcionamiento del renderer (usando Spec) y el cuestionario generado, usando Mocha, Chai y Karma.

%++++++++++++++++++++++++++++++++++++++++++++++++++++++++++++++++++++++++++++++
\section{Creaci\'on del renderer Sinatra}
\label{:sec3}

Este otro renderer genera una aplicaci\'on Sinatra con todo lo necesario para ser desplegada en {\bfseries Heroku} o ejecutar localmente. 
\bigskip

\begin{enumerate}
  \item Esta aplicaci\'on guardar\'a todos los datos del cuestionario y de los alumnos permitidos para realizarlo en una hoja de c\'alculo de {\bfseries Google Drive}
  en la cuenta del profesor.

\item Posteriormente, servir\'a el cuestionario a los alumnos especificados (durante un periodo de tiempo establecido previamente). Una vez que los alumnos hayan
  completado el cuestionario, sus respuestas se guardar\'an en una copia del cuestionario solo visible para los profesores a trav\'es de Google Drive y su
  nota se guardar\'a adem\'as en la hoja de c\'alculo de Google Drive. De este modo, quedar\'a constancia de la realizaci\'on del mismo.
  \bigskip

\item Los alumnos podr\'an reintentar el cuestionario todas las veces que deseen mientras se encuentre activo.

\end{enumerate}

{\bfseries NOTA}: Es responsabilidad del profesor facilitar la nota a los alumnos en el momento que estime oportuno.
\bigskip

En el fichero del examen que recibe como entrada la gema, adem\'as de definir las preguntas, se debe especificar a los usuarios que har\'an
uso de la aplicaci\'on:

\begin{itemize}
  \item Por un lado, se deben indicar a los profesores que podr\'an desplegar el examen o consultar las notas de los alumnos. Se indicar\'a
  su email de Google en forma de \textit{string}. En caso de ser m\'ultiples profesores, se usar\'a una notaci\'on de \textit{array}.
  
  [foto]
  
  \item Por otra parte, se deber\'an indicar los alumnos permitidos para realizar el cuestionario. Se puede usar un \textit{Hash} con 
  la informaci\'on necesaria de ellos, o indicar el path de un fichero {\bfseries CSV} con los datos de los mismos.
  
  [foto]
  \bigskip
  
  El formato del fichero CSV debe ser del siguiente modo:
  
  [tabla]
  
\end{itemize}

Para resolver el gran problema de la autentificaci\'on de usuarios, se hace uso de OAuth. De este modo, delegamos todo el servicio a Google y evitamos, por tanto,
posibles brechas de seguridad que den lugar a suplantaciones de identidad o exposici\'on de datos sensibles de los usuarios a terceras personas.
\bigskip

Adem\'as, es necesario especificar un fichero \textit{config.yml} que contiene la ventana temporal en la cual estar\'a disponible
el cuestionario, el nombre del subdominio de Heroku que se desea usar para desplegar el cuestionario y la informaci\'on relativa a Google Drive:
\begin{itemize}
  \item Nombre de la hoja de c\'alculo donde se guardar\'an los datos de alumnos y preguntas y respuestas.
  \item Nombre de la carpeta que contendr\'a dicha hoja de c\'alculo.
  \item Path donde queremos que se cree la carpeta y la hoja de c\'alculo (si no existe alguna carpeta del path, se crear\'a tambi\'en).
  \item API keys necesarias para poder usar los servicios de Google, tanto la autenticaci\'on con OAuth como la escritura en Google Drive.
\end{itemize}

[foto de config.yml]
\bigskip

Finalmente, los ficheros que genera este renderer son:
\begin{itemize}
  \item El c\'odigo Ruby del servidor.
  \item Las vistas necesarias (incluyendo el cuestionario generado en HTML y un template ERB que se usar\'a para crear las copias de los cuestionarios
  realizados por los alumnos).
  \item Un Gemfile con las dependencias necesarias.
  \item Un Rakefile para automatizar tareas (de ejecuci\'on y despliegue de la aplicaci\'on).
  \item Una carpeta denominada \textit{config} con los datos de alumnos, profesores, las preguntas y respuestas y una copia del fichero \textit{config.yml}
  del cual se hara una lectura de los par\'ametros. De este modo, evitamos que las variables existentes en el c\'odigo contengan la informaci\'on sensible.
\end{itemize}

[foto del arbol de directorios y ficheros]

