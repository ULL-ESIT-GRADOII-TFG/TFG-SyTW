%%%%%%%%%%%%%%%%%%%%%%%%%%%%%%%%%%%%%%%%%%%%%%%%%%%%%%%%%%%%%%%%%%%%%%%%%%%%%%%
% Chapter 3: Mejoras del DSL original
%%%%%%%%%%%%%%%%%%%%%%%%%%%%%%%%%%%%%%%%%%%%%%%%%%%%%%%%%%%%%%%%%%%%%%%%%%%%%%%

%++++++++++++++++++++++++++++++++++++++++++++++++++++++++++++++++++++++++++++++

Fruto del estudio del c\'odigo y de ejecuciones sucesivas de la \ceit{gema} se detectaron una serie de errores de funcionamiento,
por lo que antes de implementar mis mejoras propias era conveniente solucionar los problemas existentes.
Adem\'as, se a\~{n}adieron algunos cambios para mejorar el funcionamiento de la gema. A continuaci\'on se detallan las mejoras
realizadas:

\begin{itemize}
  \item Correcci\'on de errores en el funcionamiento de la gema. Las opciones enumeradas a continuaci\'on no funcionaban correctamente:
  \begin{itemize}
    \item La opci\'on que permite indicar si el orden de las respuestas en las preguntas de completar espacios en blaco importa o no.
    \item La opci\'on de a\~{n}adir comentarios opcionales a los textos de las preguntas.
    \item La opci\'on \textit{raw} que permite incrustar el texto de las preguntas entre etiquetas \textless pre\textgreater \space \ceit{HTML}.
    \item La opci\'on de explicaci\'on global para todos los \textit{distractors} no funcionaba.
  \end{itemize}
  \item \ceit{Refactorizaci\'on} de c\'odigo, evitando la repetici\'on del mismo en la medida de lo posible.
  \item A\~{n}adido manejo de excepciones tras errores de ejecuci\'on.
  \item A\~{n}adida la opci\'on en l\'inea de comandos \textit{--version} para comprobar la versi\'on de la gema. 
  \begin{verbatim}
  [~/tmp]$ ruql --version
  \end{verbatim}
  \newpage
  
  \item A\~{n}adida la opci\'on en l\'inea de comandos \textit{--help} para ver la ayuda. 
  \begin{verbatim}
  [~/tmp]$ ruql --help
  \end{verbatim}
\end{itemize}
\bigskip

%---------------------------------------------------------------------------------
\section{Problemas encontrados y soluciones}
\label{3:sec:1}

A continuaci\'on se detallan los problemas encontrados durante la implementaci\'on de las mejoras del \ceit{DSL} original y las soluciones
encontradas para los mismos.

\subsection{Entender el funcionamiento del c\'odigo de la gema}
\label{subsec:3.1.1}
\bigskip

{\normalsize {\bfseries Soluci\'on}}
\bigskip

Leer la documentaci\'on de la gema, generar cuestionarios de pruebas y estudiar el c\'odigo fuente.

\subsection{Corregir tests y funcionalidades de la gema}
\label{subsec:3.1.2}
\bigskip

{\normalsize {\bfseries Soluci\'on}}
\bigskip

Tras realizar el correspondiente \cei{\ref{apend1:fork}} en \ceit{GitHub} para empezar a implementar mis modificaciones, ejecut\'e los tests de la 
gema original para comprobar la ausencia de fallos. Al finalizar, algunos tests fallaron por lo que decid\'{\i} corregirlos. Del 
mismo modo, algunas gemas de \ceit{testing} existentes en el \ceit{\ref{apend1:gemfile}} presentaban incompatibilidades con las nuevas versiones de \ceit{Ruby}, por
lo que tambi\'en se corrigi\'o.
\bigskip
