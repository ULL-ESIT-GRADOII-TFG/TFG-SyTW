%%%%%%%%%%%%%%%%%%%%%%%%%%%%%%%%%%%%%%%%%%%%%%%%%%%%%%%%%%%%%%%%%%%%%%%%%%%%%%%
% Chapter 2: Desarrollo
%%%%%%%%%%%%%%%%%%%%%%%%%%%%%%%%%%%%%%%%%%%%%%%%%%%%%%%%%%%%%%%%%%%%%%%%%%%%%%%

%++++++++++++++++++++++++++++++++++++++++++++++++++++++++++++++++++++++++++++++

En el cap\'{\i}tulo anterior se ha definido el Trabajo de Fin de Grado, especificado los objetivos y actividades
a desarrollar y mencionado las tecnolog\'{\i}as empleadas para su desarrollo. A continuaci\'on, se describir\'a 
la metodolog\'{\i}a de trabajo seguida y los problemas hallados durante el desarrollo junto con las soluciones
alcanzadas.

%++++++++++++++++++++++++++++++++++++++++++++++++++++++++++++++++++++++++++++++

\section{Metodolog\'{\i}a usada}
\label{sec:1}

Se ha llevado a cabo una metodolog\'{\i}a de trabajo \textit{\'agil}, com\'un en el campo de la Ingenier\'{\i}a Inform\'atica,
con iteraciones semanales en las que se defin\'{\i}an una serie de tareas u objetivos y que se presentaban la siguiente semana. 
De este modo, con la entrega de prototipos funcionales de la aplicaci\'on, se han ido testeando, corrigiendo y mejorando las 
funcionalidades, al mismo tiempo que detectando problemas no contemplados en las fases previas de dise�o. 
\bigskip

Esta metodolog\'{\i}a, adem\'as, ha propiciado la generaci\'on de ideas que se han traducido en nuevas caracter\'{\i}sticas.

\bigskip
git

issues

ramas

experiencia en asignaturas

reuniones

...

\bigskip

objetivos

intro a objetivos


capitulo por renderer con el objetivo, desarrollo y resultados, problemas

%---------------------------------------------------------------------------------
\section{Problemas encontrados y soluciones}
\label{sec:2}

A continuaci\'on se detallan los problemas encontrados durante la implementaci\'on del Trabajo de Fin de Grado y las soluciones
encontradas para los mismos.

\subsection{Entender el funcionamiento del c\'odigo de la gema}
\label{subsec:2.1}
\bigskip

{\normalsize {\bfseries Soluci\'on}}

Leer la documentaci\'on de la gema, generar cuestionarios de pruebas y estudiar el c\'odigo fuente.

\subsection{Corregir tests y funcionalidades de la gema}
\label{subsec:2.2}
\bigskip

{\normalsize {\bfseries Soluci\'on}}

Tras realizar el correspondiente \textit{fork} en GitHub para empezar a implementar mis modificaciones, ejecut\'e los tests de la 
gema original para comprobar la ausencia de fallos. Al finalizar, algunos tests fallaron por lo que decid\'{\i} corregirlos. Del 
mismo modo, algunas gemas de testing existentes en el Gemfile presentaban incompatibilidades con las nuevas versiones de Ruby, por
lo que tambi\'en se corrig\'{\i}o.
\bigskip

Del mismo modo, las siguientes funcionalidades de la gema fueron corregidas ya que no funcionaban correctamente:
\begin{itemize}
  \item La opci\'on que permite indicar si el orden de las respuestas en las preguntas de completar espacios en blaco importa o no.
  \item La opci\'on de a�adir comentarios opcionales a los textos de las preguntas.
  \item La opci\'on \textit{raw} que permite incrustar el texto de las preguntas entre etiquetas \textless pre\textgreater \space HTML.
  \item La opci\'on de explicaci\'on global para todos los \textit{distractors} no funcionaba.
\end{itemize}

\subsection{Correcci\'on de preguntas de Ruby en JavaScript}
\label{subsec:2.3}
\bigskip

bla, bla, bla

\subsection{Correcci\'on de preguntas de JavaScript en Ruby}
\label{subsec:2.3}
\bigskip

bla, bla, bla

\subsection{Timeout corto entre peticiones del navegador al servidor}
\label{subsec:2.4}
\bigskip

bla, bla, bla

\subsection{Problema de seguridad al evaluar c\'odigo Ruby en el servidor}
\label{subsec:2.5}
\bigskip

bla, bla, bla

\subsection{Lugar de almacenamiento de las respuestas de los alumnos}
\label{subsec:2.5}
\bigskip

La idea principal era almacenar todas las preguntas y respuestas de los alumnos en una base de datos pero, viendo la evoluci\'on que ha tenido la
herramienta Google Drive y el aumento considerable de su uso por parte de docentes, decid\'{\i} sustituir las tradicionales y siempre mon\'otonas consultas a bases de datos por esta 
herramienta de almacenamiento que permite visualizar y administrar f\'acilmente toda la informaci\'on.