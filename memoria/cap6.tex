%%%%%%%%%%%%%%%%%%%%%%%%%%%%%%%%%%%%%%%%%%%%%%%%%%%%%%%%%%%%%%%%%%%%%%%%%%%%%
% Chapter 5: Conclusiones y Trabajos Futuros 
%%%%%%%%%%%%%%%%%%%%%%%%%%%%%%%%%%%%%%%%%%%%%%%%%%%%%%%%%%%%%%%%%%%%%%%%%%%%%%%

%++++++++++++++++++++++++++++++++++++++++++++++++++++++++++++++++++++++++++++++

Desde hace unos a�os hasta ahora, ha tenido lugar un enorme crecimiento de las \ceit{plataformas} de aprendizaje online. Cada vez
cuentan con m\'as adeptos y las instituciones de ense\~{n}anza saben que incorporarlas a sus sistemas educativos es clave
para ofrecer un servicio puntero y de calidad.

\'Esto es lo que se pretende con la herramienta obtenida tras la realizaci\'on de este Trabajo de Fin de Grado: que sea posible
su implantaci\'on dentro del marco acad\'emico de la Universidad de La Laguna, partiendo de la premisa de que nos estamos adentrando 
en una \'epoca en la que estas herramientas de aprendizaje online seguir\'an evolucionando y teniendo un papel importante en la educaci\'on.
\bigskip

En este trabajo se proporciona/extiende un \ceis{Lenguaje de Dominio Espec\'{\i}fico} (\cei{DSL}) para la \ceit{elaboraci\'on} y \ceit{evaluaci�n} de \ceit{cuestionarios} que ofrece varias ventajas 
con respecto a otras herramientas equivalentes:
\begin{enumerate}
\item Respuestas que pueden ser evaluadas mediante \ceit{expresiones regulares} extendidas.
\item Respuestas que pueden ser evaluadas mediante \ceit{c\'odigo} arbitrario definido por el profesor.
\item Generaci\'on de c\'odigo para un n\'umero amplio de plataformas y formas de uso: generaci�n de \ceit{HTML} \cei{standalone} para retroalimentaci�n y entrenamiento del alumno, \ceit{XML} para 
\ceit{EdX}, etc.
\item \ceit{Escalabilidad}: posibilidad de implantar soporte para cualquier plataforma o forma de uso que se desee mediante el uso de \ceit{renderers}.
\item Generaci\'on de una \ceit{aplicaci\'on} de \ceit{evaluaci\'on} completa.
\item XXXX aumentar como quieras
\end{enumerate}

Los objetivos marcados al comienzo de la asignatura han sido cumplidos, adem\'as de los objetivos generales establecidos en la gu\'{\i}a
docente de la asignatura y la adquisi\'on de las competencias generales y espec\'{\i}ficas descritas en el mismo. Sin embargo, el desarrollo 
no finaliza aqu\'{\i}. A\'un queda trabajo por realizar para poder garantizar un eficaz funcionamiento y rendimiento de esta herramienta. 
El estado final de la misma puede ser el punto de partida para pr\'oximos Trabajos de Fin de Grado.
\bigskip

As\'{\i} pues, las principales l\'{\i}neas de desarrollo a continuar ser\'{\i}an las enumeradas a continuaci\'on:

\begin{itemize}
  \item Resolver los problemas de \ceit{seguridad} relacionados con evaluar el c\'odigo escrito por los alumnos.
  \item Dar soporte a preguntas con respuestas de c\'odigo en otros \ceit{lenguajes de programaci\'on}.
  \item Ofrecer una alternativa de despliegue distinta a \ceit{Heroku}, como podr\'{\i}a ser un \ceit{servidor dedicado} ofrecido por la Universidad de
  La Laguna.
\end{itemize}
