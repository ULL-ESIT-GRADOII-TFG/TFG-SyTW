%%%%%%%%%%%%%%%%%%%%%%%%%%%%%%%%%%%%%%%%%%%%%%%%%%%%%%%%%%%%%%%%%%%%%%%%%%%%%
% Chapter 5: Conclusiones y Trabajos Futuros 
%%%%%%%%%%%%%%%%%%%%%%%%%%%%%%%%%%%%%%%%%%%%%%%%%%%%%%%%%%%%%%%%%%%%%%%%%%%%%%%

%++++++++++++++++++++++++++++++++++++++++++++++++++++++++++++++++++++++++++++++

Desde hace unos a�os hasta ahora, ha tenido lugar un enorme crecimiento de las \ceit{plataformas} de aprendizaje online. Cada vez
cuentan con m\'as adeptos y las instituciones de ense\~{n}anza saben que incorporarlas a sus sistemas educativos es clave
para ofrecer un servicio puntero y de calidad.

\'Esto es lo que se pretende con la herramienta obtenida tras la realizaci\'on de este Trabajo de Fin de Grado: que sea posible
su implantaci\'on dentro del marco acad\'emico de la Universidad de La Laguna, partiendo de la premisa de que nos estamos adentrando 
en una \'epoca en la que estas herramientas de aprendizaje online seguir\'an evolucionando y teniendo un papel importante en la educaci\'on.
\bigskip

En este trabajo se proporciona/extiende un \ceis{Lenguaje de Dominio Espec\'{\i}fico} (\cei{DSL}) para la \ceit{elaboraci\'on} y \ceit{evaluaci�n} de \ceit{cuestionarios} que ofrece varias ventajas 
con respecto a otras herramientas equivalentes:
\begin{enumerate}
  \item Portabilidad: la principal ventaja sobre plataformas como Moodle\cite{moodle}.
  \item Respuestas que pueden ser evaluadas mediante \ceit{expresiones regulares} extendidas.
  \item Respuestas que pueden ser evaluadas mediante \ceit{c\'odigo} arbitrario definido por el profesor.
  \item Generaci\'on de c\'odigo para diversas plataformas y formas de uso: generaci\'on de \ceit{HTML} \cei{standalone} para entrenamiento y retroalimentaci\'on del alumno, \ceit{XML} 
  para \ceit{edX}\cite{edx} y \ceit{AutoQCM} para \ceit{AMC}\cite{amc}.
  \item \ceit{Escalabilidad}: posibilidad de implantar soporte para cualquier otra plataforma o forma de uso que se desee mediante el uso de \ceit{renderers}.
  \item Generaci\'on de una \ceit{aplicaci\'on} \ceit{Sinatra} de \ceit{evaluaci\'on} completa con almacenamiento de ex\'amenes y datos en \ceis{Google Drive}. Una soluci\'on innovadora que
  facilita la gesti\'on de los mismo y evita tener que almacenar preguntas, respuestas, alumnos y calificaciones en bases de datos tradicionales junto con los problemas relacionados con su 
  manipulaci\'on y exportaci\'on a formatos m\'as \'utiles, como puede ser, las hojas de c\'alculo.
\end{enumerate}

Adem\'as, teniendo en cuenta aspectos \'eticos y legales, se hace uso de \ceis{OAuth} para delegar a \ceit{Google} la acci\'on de la autentificaci\'on. De este modo, se evitan problemas de brechas de 
seguridad como puede ser la suplantaci\'on de identidad o la exposici\'on de datos sensibles de los usuarios a terceras personas.
\bigskip

Para concluir, podemos afirmar que los objetivos han sido cumplidos, tanto los establecidos de la gu\'{\i}a docente de la asignatura junto con la adquisi\'on de las competencias generales y 
espec\'{\i}ficas como los marcados al comienzo de la asignatura. Sin embargo, el desarrollo no finaliza aqu\'{\i}. A\'un queda trabajo por realizar para poder garantizar un eficaz funcionamiento y 
rendimiento de esta herramienta. El estado final de la misma puede ser el punto de partida para pr\'oximos Trabajos de Fin de Grado.
\bigskip

As\'{\i} pues, las principales l\'{\i}neas de desarrollo a continuar ser\'{\i}an las enumeradas a continuaci\'on:
\begin{itemize}
  \item Resolver los problemas de \ceit{seguridad} relacionados con evaluar el c\'odigo escrito por los alumnos.
  \item Dar soporte a preguntas con respuestas de c\'odigo en otros \ceit{lenguajes de programaci\'on}.
  \item Ofrecer una alternativa de despliegue distinta a \ceit{Heroku}, como podr\'{\i}a ser un \ceit{servidor dedicado} ofrecido por la Universidad de
  La Laguna.
\end{itemize}
